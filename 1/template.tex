\documentclass[a4paper, 12pt]{article}
\include{includes/packages}
\newcommand{\Title}{Отчет о выполнении лабораторной работы}
\newcommand{\TaskType}{лабораторная работа №1}
\newcommand{\SubTitle}{по дисциплине <<Интеллектуальные технологии и системы>>}
\newcommand{\LabTitle}{Исследование инструментальных средств и технологий
разработки интеллектуальных систем} 
\newcommand{\Faculty}{<<Информатика и системы управления>>}
\newcommand{\Department}{<<Компьютерные системы и сети (ИУ-6)>>}
\newcommand{\AuthorFull}{Козлов Владимир Михайлович}
\newcommand{\Author}{Козлов В.М.}
\newcommand{\Teacher}{Пугачев Е.К.}
\newcommand{\group}{ИУ6-13М}
\newcommand{\Year}{2024}
\newcommand{\Country}{Россия}
\newcommand{\City}{Москва}

\newcommand{\UpperFullOrganisationName}{Министерство науки и высшего образования Российской Федерации}
\newcommand{\ShortOrganisationName}{МГТУ~им.~Н.Э.~Баумана}
\newcommand{\FullOrganisationName}{федеральное государственное бюджетное образовательное учреждение высшего профессионального образования\newline <<Московский государственный технический университет имени Н.Э.~Баумана (национальный исследовательский университет)>> (\ShortOrganisationName)}

\include{includes/styles}
\begin{document}
\input{includes/titlepage}
\pagebreak
\tableofcontents
\newpage
% Основная часть --------------------------------------------------------------------------------------------
\section*{Цель}
\addcontentsline{toc}{section}{Цель}
Приобретение навыков проектирования и реализации основных
элементов систем искусственного интеллекта
\section*{Задание}
\addcontentsline{toc}{section}{Задание}
Дана база фактов элементов электрической принципиальной схемы. Написать программу, которая на основе фактов базы выводит на экран схему и выдает комментарии по каждому элементу.
\newpage
% -------------------------------------------------------
\section{Структурная карта Константайна}
\begin{center}
  \centering
  \includegraphics[width=.7\linewidth]{extra/constantine.png}
  \captionof{figure}{Структурная карта Константайна}
  \label{fig:prplot}
\end{center}
\section{Программа}
\subsection{Код программы}
%\lstinputlisting[caption=Листинг программы]{extra/LAB.PRO}
\subsection{Результат выполнения программы}
\newpage
%---------------------------------------------------------
\section{Вывод}
В ходе данной лабораторной работы осуществлено
ознакомление с базовыми функциями и принципами использования языка
Turbo Prolog. Было осуществлено проектирование структуры и реализация
программы для решения, поставленной задачи на данном языке.
\end{document}
